\section*{Введение}
\addcontentsline{toc}{section}{Введение}

Тема данной курсовой работы актуальна, так как тригонометрические функции активно используются в неевклидовой геометрии, в частности - геометрии Лобачевского, и частично используются в активно развивающейся сфере искуственного интеллекта.

Сам феномен гиперболических функций задокументированно существует уже более трех сотен лет.

Проблема исследования заключается в том, что обычно с гиперболическими функциями обучающиеся ВУЗов не профильных направлений знакомятся очень редко в курсе и лишь в курсе дифференциальной геометрии. Такой подход к изучению данной темы ошибочен. Изучение темы "Гиперболические функции" позволяет обучающимся расширить и улучшить свои навыки дифференцирования и интегрирования различных функций.

Исходя из выявленной проблемы, определена тема курсовой работы: "Гиперболические функции".

Объект исследования: Гиперболические функции

Предмет исследования: Применение и особенности гиперболических фнкций.

Цель исследования: изучить теоретические и практические сведенья о гиперболических функциях и собрать данные в одном месте.

Для осуществления обозначенной цели служат следующие задачи:

    \begin{enumerate}
        \item Изучить научную, учебную и справочную литературу по теме исследования;
        \item Раскрыть сущность гиперболических функций;
        \item Проанализировать гиперболические уравнения от одной переменной;
        \item Привести собственные примеры некоторых задач связаных с гиперболическими функциями и их частных решений;
        \item Проанализировать практические задачи, приводящие к использованию гиперболических функций.
    \end{enumerate}

В ходе исследования были использованы следующие методы:

\begin{enumerate}
    \item Метод анализа;
    \item Метод синтеза;
    \item Метод аналогии.
\end{enumerate}

Практическая значимость данной работы состоит в возможности использования данной работы в качестве справочного материала или учебного пособия.

Работа состоит из введения, 2 глав, 9 параграфов, заключения, списка использованной литературы.

Во введении проведено обоснование актуальности данной темы, определена проблема исследования, объект и предмет, цели и задачи, методы исследования, а также выделена его практическая значимость.

Параграфы 1-7 (1 глава) раскрывают теоретические аспекты нашей работы. Параграфы 8-9 (2 глава) содержат практические задачи связаные с гиперболическими функциями и информацию об применении гиперболических функций в современном мире.

В заключении обобщаются результаты курсовой работы, оцениваются достижения поставленной цели исследования и полнота решения задач, а также формулируются основные выводы


\newpage
\subsection*{Глава I. Теория гиперболических функций}
\addcontentsline{toc}{section}{Глава I. Теория гиперболических функций}

\subsection*{1.1 История гиперболических функций}
\addcontentsline{toc}{subsection}{1.1 История гиперболических функций}

Первое появление гиперболических функций историки математики обнаружили (1707, 1722) в
трудах английского математика, ученика и помощника И. Ньютона, Абрахама де Муавра (Abraham
de Moivre, 1667–1754). Дал современное определение и выполнил обстоятельное исследование этих функций Винченцо Риккати в 1757 году; он
же предложил обозначения для них: sh для гиперболического синуса и ch для гиперболического
косинуса (в западной нотации sinh и cosh). Риккати исходил из рассмотрения единичной гиперболы, используя аналогию с единичной окружностью для тригонометрических функций. Независимое
открытие и дальнейшее исследование свойств гиперболических функций было проведено Иоганном
Ламбертом, который установил широкий параллелизм формул обычной и гиперболической тригонометрии (1770). Н.И. Лобачевский (1792–1856) впоследствии использовал этот параллелизм, доказывая непротиворечивость неевклидовой геометрии, в которой круговая тригонометрия заменяется
на гиперболическую. Однако все это случилось много позже знаменитого, говоря современным языком, «баттла» трех ведущих математиков своего времени (Иоганн Бернулли, Готфрид Лейбниц и
Кристофер Гюйгенс), решивших поставленную Якобом Бернулли в 1690 году задачу об определении
формы цепной линии (гиперболический косинус). Вероятно, именно это состязание способствовало
пробуждению осознанного интереса к гиперболическим функциям.

Однако за почти два столетия до начала осознанного изучения гиперболических функций Гиртом
де Крёмером (Gheert de Kr¨еmer, 1512–1594) было обнаружено первое и одно из наиболее важных
применений гиперболических функций. Его осуществил фламандский географ, более известный под
латинским именем Герард (Джерард) Меркатор (Gerhardus Mercator).

\subsection*{1.2 Гиперболические функции}
\addcontentsline{toc}{subsection}{1.2 Гиперболические функции}
Гиперболическая функция - это функция от одной переменной, выражающаяся через экспоненту и связанная с тригонометрической функцийей. Существует 6 видов таких функций: 
гиперболический синус $sh(x)$ (аналог для английской литературы $sinh(x)$), гиперболический косинус $ch(x)$ (аналог для английской литературы $cosh(x)$), 
гиперболический тангенс $th(x)$ (аналог для английской литературы $tanh(x)$), гиперболический котангенс $cth(x)$ (аналог для английской литературы $csch(x)$), 
гиперболический секанс $sch(c)$ (аналог для английской литературы $sech(x)$), гиперболический косеканс $csch(x)$ (аналог для английской литературы $coth(x)$).

\begin{equation}
    sh(x)=\frac{e^{ix}-e^{-ix}}{2}
 \end{equation}
 
 \begin{equation}
    ch(x)=\frac{e^{ix}+e^{-ix}}{2}
 \end{equation}
 
 \begin{equation}
    th(x)=\frac{e^{ix}-e^{-ix}}{e^{ix}+e^{-ix}}=\frac{e^{2x}-1}{e^{2x}+1}
 \end{equation}
 
 \begin{equation}
    cth(x)=\frac{e^{ix}+e^{-ix}}{e^{ix}-e^{-ix}}=\frac{e^{2x}+1}{e^{2x}-1}
 \end{equation}
 
 \begin{equation}
    sch(x)=\frac{2}{e^{x}+e^{-x}}
 \end{equation}
 
 \begin{equation}
    csch(x)=\frac{2}{e^{x}-e^{-x}}
 \end{equation}
 
Разница между гиперболическими функциями и тригонометрическими функциями заключается в наличии мнимой части в уравнении. 
К примеру функции тригонометрического косинуса $(7)$ и гиперболического косинуса $(8)$ отличаются наличием $i$ (мнимой части) в уравнении тригонометрического косинуса и, соответственно, её отсутствиии в уравнении гиперболического косинуса:\\
 \begin{equation}
    \cos(x)=\frac{1}{2}*(e^{ix}+e^{-ix})
 \end{equation}
 \begin{equation}
    ch(x)=\frac{1}{2}*(e^{x}+e^{-x})
\end{equation}

\subsection*{1.3 Задание гиперболических функций}
\addcontentsline{toc}{subsection}{1.3  Задание гиперболических функций}

Гиперболические функции являются функциями, котороые могут быть представленны через степенные экспонециальные функции $e^x$ и $e^{-x}$. 

Рассмотрим как графики гиперболического синуса и гиперболического косинуса формируются при помощи экспоненциальных функций. 

\image{image5.png}{Формирование графиков гиперболического синуса и гиперболического косинуса при помощи экспоненциальных функций}{1}

На рисунке 0.1. можно увидеть как график $y = sh(x)$ (слева) и $y=ch(x)$ (справа) собраны при помощи экспоненциальных функций, а именно:

\begin{itemize}
    \item Гиперболический синус через графическое сложение функций $y=\frac{1}{2}e^x$ и $y=-\frac{1}{2}e^{-x}$
    \item Гиперболический косинус через графическое сложение функций $y=\frac{1}{2}e^x$ и $y=\frac{1}{2}e^{-x}$
\end{itemize}

Следовательно мы можем сделать следующие преобразования и вывести формулы для гиперболического синуса(9) и гиперболического косинуса(10):
\begin{equation}
    sh(x) = \frac{1}{2}e^x - \frac{1}{2}e^{-x} = \frac{e^x-e^{-x}}{2}
 \end{equation}

 \begin{equation}
    ch(x) = \frac{1}{2}e^x + \frac{1}{2}e^{-x} = \frac{e^x+e^{-x}}{2}
 \end{equation}

 На основании полученых формул получим формулу гиперболического тангенса(11):

 \begin{equation}
    th(x) = \frac{\frac{e^x-e^{-x}}{2}}{\frac{e^x+e^{-x}}{2}} = \frac{e^x-e^{-x}}{e^x+e^{-x}}
 \end{equation}

Следует помнить, что в случае гиперболического косинуса, равного нулю, гиперболический тангенс не существует.

Соответственно получим формулы для обратных (ареа) функций:

\begin{itemize}
    \item Гиперболического арксинуса - $arsh(x) = \frac{1}{sh(x)} = \frac{1}{\frac{e^x-e^{-x}}{2}} = \frac{2}{e^x-e^{-x}}$
    \item Гиперболического арккосинуса - $arch(x) = \frac{1}{ch(x)} = \frac{1}{\frac{e^x+e^{-x}}{2}} = \frac{2}{e^x+e^{-x}}$
\end{itemize}
\subsection*{1.4 Свойства гиперболических функций}
\addcontentsline{toc}{subsection}{1.4 Свойства гиперболических функций}
Гиперболические функции, также как и тригонометрические, можно использовать для параметрического задания различных кривых. Например через синус и косинус задается эллипс (12), а через гиперболический синус и гиперболический косинус - гипербола (13).
\begin{equation}
    \begin{cases}
        x = a * \cos(t)\\
        y = a * \sin(t)\\
    \end{cases}
\end{equation}

\begin{equation}
    \begin{cases}
        x = a * ch(t)\\
        y = a * sh(t)\\
    \end{cases}
\end{equation}

Сущестуют гиперболические тождества аналогичные тригонометрическим тождествам (14) (15) (16), формулы двойного угла для гиперболического синуса (17) и гиперболического косинуса (18), а также формулы половинного угла для гиперболического тангенса (19) и гиперболического котангенса (20).
\begin{equation}
    ch(t)^2 - sh(t)^2 = 1
\end{equation}
\begin{equation}
    ch(t) + sh(t) = e^t
\end{equation}
\begin{equation}
    ch(t) - sh(t) = -e^{-t}
\end{equation}
\begin{equation}
    sh(2t) = 2sh(t)*ch(t)
\end{equation}
\begin{equation}
    ch(2t) = 2ch(t)^2 - 1 = 1 + 2sh(t)^2
\end{equation}
\begin{equation}
    th(\frac{z}{2})=\frac{sh(x)+i*\sin(y)}{ch(x)-\cos(y)}
\end{equation}
\begin{equation}
    cth(\frac{z}{2})=\frac{sh(x)-i*\sin(y)}{ch(x)-\cos(y)}
\end{equation}

Гиперболические функции также делятся на четные и не четные (Таблица 1).

Таблица 1. Четность и не четность гиперболических функций.
\begin{center}
\begin{tabular}{| p{3cm} | p{3cm} | p{3cm} | p{3cm} |}
    \hline
    Функция от x & Функция от -x & Аналог для функции от -x при x & Четность функции \\ \hline
    $ch(x)$ & $ch(-x)$ & $ch(x)$ & Четная  \\ \hline
    $sh(x)$ & $sh(-x)$ & $-sh(x)$ & Не четная \\\hline
    $th(x)$ & $th(-x)$ & $-th(x)$ &  Не четная  \\ \hline
    $cth(x)$ & $cth(-x)$ & $-cth(x)$ &  Не четная \\ \hline
    $sch(x)$ & $sch(-x)$ & $sch(x)$ & Четная \\ \hline
    $csch(x)$ & $csch(-x)$ & $-csch(x)$ &  Не четная \\ \hline
    \hline
\end{tabular}
\end{center}


\subsection*{1.5 Обратные гиперболические функции}
\addcontentsline{toc}{subsection}{1.5 Обратные гиперболические функции}

Обратные гиперболические функции также известные как ареа-функции - семейство элементарных функций, обпределяющихся как обратные функции для гиперболических функций. В отличии от обратных тригонометрических функций, которые обозначают длинну дуги единичной окружности, обратные гиперболические функции обозначают площадь сектора единичной параболы.

Существуют функции: ареасинус $arsh(x)$, ареакосинус $arch(x)$, ареатангенс $arth(x)$, ареакотангенс $arcth(x)$, ареасеканс $arsch(x)$ и ареакосеканс $arcsch(x)$.

\subsection*{1.6 Дифференцирование и интегрирование гиперболических функций}
\addcontentsline{toc}{subsection}{1.6 Дифференцирование и интегрирование гиперболических функций}
Как и тригонометрические функции, гиперболические функции подвержены дифференцированию и интегрированию.

Таблица 2. Дифференцирование гиперболических функций.
\begin{center}
    \begin{tabular}{| p{3cm} | p{7cm} | p{3cm} |}
        \hline
        Функция от x & Результат дифференцирования & Ограничения\\ \hline
        $\frac{d}{dx}sh(x)$ & $ch(x)$ & \\ \hline
        $\frac{d}{dx}ch(x)$ & $sh(x)$ & \\ \hline
        $\frac{d}{dx}th(x)$ & $1-th(x)^2=sch(x)^2=\frac{1}{csh(x)^2}$ & \\ \hline
        $\frac{d}{dx}cth(x)$ & $1-cth(x)^2=-csch(x)^2=-\frac{1}{sh(x)^2}$ & $x\neq0$\\ \hline
        $\frac{d}{dx}sch(x)$ & $-th(x)sch(x)$ & \\ \hline
        $\frac{d}{dx}csch(x)$ & $-cth(x)csch(x)$ & $x\neq0$\\ \hline
    \end{tabular}
\end{center}

Таблица 3. Стандарнтые интегралы от гиперболических функций.
\begin{center}
    \begin{tabular}{| p{5cm} | p{7cm} |}
        \hline
        Интеграл & Результат интегрирования\\ \hline
       $\int sh(ax) dx$ & $a^{-1}ch(ax)+C$\\ \hline
       $\int ch(ax) dx$ & $a^{-1}sh(ax)+C$\\ \hline
       $\int th(ax) dx$ & $a^{-1}ln(ch(ax))+C$\\ \hline
       $\int cth(ax) dx$ & $a^{-1}ln|sh(ax)|+C$\\ \hline
       $\int sch(ax) dx$ & $a^{-1}atan(sh(ax))+C$\\ \hline
        & $a^{-1}ln|th(\frac{ax}{2})|+C$\\ 
       $\int csch(ax) dx$ & $a^{-1}ln|cth(ax)-chcs(ax)|+C$\\ 
        & $-a^{-1}arcth(ch(ax))+C$\\ \hline
    \end{tabular}
\end{center}

\subsection*{1.7 Графики гиперболических функций}
\addcontentsline{toc}{subsection}{1.7 Графики гиперболических функций}
Графики функций предоставляют собой кривые, схожие по своему виду с кривыми второго порядка.

График гиперболического синуса схож с кубической параболой, а график гиперболического косинуса - с квадратичной параболой. Обе эти параболы зависят от $x$. (Рис. 0.2.) 
\image{image1.png}{$y=sinh(x)$ и $y=cosh(x)$}{0.75}

График гиперболического тангенса отдаленно схож с гиперболой зависимой от $x$, а котангенс в свою очередь с кубической парасболой зависимой от $y$. (Рис. 0.3.)
\image{image2.png}{$y=th(x)$ и $y=cth(x)$}{0.75}

График секанса представляет собой %TODO% >>>>>>>>>>>>>>>>>>>>>>>>>>>>>>>>>>
леминискаты Бернулли, а грфик косеканса как и график гиперболического тангенса представляет собой гиперболу зависимую от $x$, но максимально приближеную к каноническому ее виду. (Рис. 0.4.)
\image{image3.png}{$y=sch(x)$ и $y=csch(x)$}{0.75}



\newpage
\subsection*{Глава II. Практическое применение гиперболических функций}
\addcontentsline{toc}{section}{Глава II. Практическое применение гиперболических функций}
\subsection*{2.1 Решение избранных задач содержащих гиперболические функции}
\addcontentsline{toc}{subsection}{2.1 Решение избранных задач содержащих гиперболические функции}
Задача 1. Вычислить определенный интеграл $\int_{0}^{10} (sh(x)*ch(x)) \,dx$.

Решение:

1. Вычислим непределенный интеграл $\int (sh(x)*ch(x)), dx$:\\
\begin{center}
    $\int (sh(x)*ch(x)), dx = \frac{ch(x)^2}{2}$
\end{center}

2. При помощи формулы Ньютона-Лейбница вычислим требуемое значение:\\
\begin{center}
    $\int_{0}^{10} (sh(x)*ch(x)) \,dx = \frac{ch(x)^2}{2}\vert_0^{10} = \frac{ch(10)^2}{2} - \frac{ch(0)^2}{2} = \frac{ch(10)^2}{2} - 1$
\end{center}
\subsection*{2.2 Использование гиперболических функций в нейронных сетях}
\addcontentsline{toc}{subsection}{2.2 Использование гиперболических функций в нейронных сетях}

В нейронных сетях используются различные линейные и не линейные математические функции для активации нейронов, но эффективными функциями являются лишь не линейные. Одна из таких функций - гиперболический тангенс.

Функция Tanh очень похожа на сигмовидную / логистическую функцию активации и даже имеет ту же S-образную форму с разницей в выходном диапазоне от -1 до 1. В Tanh, чем больше входные данные (более положительные), тем ближе выходное значение будет к 1.0, тогда как чем меньше входные данные (более отрицательные), тем ближе выходные данные будут к -1.0.
\image{image4.png}{Функция активации гиперболический тангенс}{1}

Преимуществами использования для активации нейронов именно данной гиперболической функции являются:
\begin{itemize}
    \item Центрированность на нуле результата обработки входных данных, следовательно, мы можем легко отобразить выходные значения как сильно отрицательные, нейтральные или сильно положительные.
    \item Обычно данная функция используется в скрытых слоях нейронной сети, поскольку его значения лежат в диапазоне от -1 до 1 и следовательно, среднее значение для скрытого слоя получается равным 0 или очень близко к нему. Это помогает в центрировании данных и значительно упрощает обучение для следующего уровня.
\end{itemize}

\newpage
\section*{Заключение}
\addcontentsline{toc}{section}{Заключение}

Итак, гипреболические функции - это аналоги тригонометрических функций из евклидовой геометрии для геометрии Лобачевского.

Самая распространная область в которой применяются гиперболические функции - это обучение нейронных сетей.

Цель исследования состояла в изучении теоретических сведений о гиперболических функциях и их систематизация.

Для достижения нашей цели были в полном объёме решены следующие задачи:
\begin{enumerate}
    \item Изучена научная, учебная и справочная литература по теме исследования;
    \item Раскрыта сущность гиперболических функций;
    \item Проанализированы гиперболические уравнения от одной пременной;
    \item Приведены собственные примеры некоторых задач связаных с гиперболическими функциями и их частных решений;
    \item Проанализированы практические задачи, приводящие к использованию гиперболических функций.
\end{enumerate}

На протяжении всей работы ма ознакомились с краткой теорией гиперболических функций, разъяснили для себя, что это такое "гиперболические функции", какими бывают задачи с использованием гиперболическимих функций, как решаются такие задачи, где используются гиперболические функции в современном мире.

Таким образом мы изучили гиперболические функции и выполнили все поставленные нами задачи данного исследования.